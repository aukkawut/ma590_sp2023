\documentclass{article}
\usepackage[utf8]{inputenc}
\usepackage[margin=1in]{geometry}
\usepackage{amsfonts,amsmath,amssymb,amsthm}
\usepackage{fancyhdr}
\usepackage{graphicx}
\usepackage{algorithm}
\usepackage{algpseudocode}
\usepackage{indentfirst}
\title{Causality with Deep Learning}
\author{Proposal\\Aukkawut Ammartayakun\\MA 590 Special Topics: Causal Inference}
\date{Spring 2023}

\begin{document}
\maketitle
%\begin{titlepage}
%   \begin{center}
%       \vspace*{1cm}
%
%       \textbf{Optimizing Information Cascade and Propagation in Social Network}
%
%       \vspace{0.5cm}
%        MA 590 Special Topics: Causal Inference
%            
%       \vspace{1.5cm}
%
%       \textbf{Aukkawut Ammartayakun}
%
%   \end{center}
%\end{titlepage}


\section{Introduction}
Traditionally, to determine the causal relationship, one must perform the randomized test to 
determine the causality of the variables. However, alternatively, one can create a causal model 
to determine the causal relationship. Both of which can be exploited with deep learning. 
However, the hypothesis testing exploitation will be explored
\subsection{Hypothesis testing exploitation}
As the process of hypothesis testing involves evaluating the test statistics and using that 
to find the critical region under the distribution. However, that can be simplified, hopefully, 
into the classification problem that can be exploited with the machine learning model. 
The transformation of the data into its representation form, i.e., embedding might reveal 
the underlying causal relation. The use of a machine learning model in substitution for 
hypothesis testing for causal inference will be explored, along with the analysis of the power 
of testing using a classification model.
\section{Dataset}
ASSISTments dataset\cite{assistment} will be explored in various aspects.
\section{Proposed Method (Roughly)}
\subsection{Hypothesis testing exploitation}
\begin{enumerate}
    \item Explore the hypothesis testing and analyze the classification problem as the 
    statistical test to show the feasibility of the method and analyze the tradeoff of this 
    method.
    \item Comparing the traditional method of determining the causal relationship with the 
    deep learning model mainly aims to classify whether to reject the null hypothesis 
    (whether A has an effect on B) or not.
\end{enumerate}

\section{Possible way}

Let say we use a Siamese network
$$L(\mathbf{x}_1, \mathbf{x}_2, y) = \frac{1}{2}(1-y)d(\mathbf{x}_1, \mathbf{x}_2)^2 + \frac{1}{2}yd(\mathbf{x}_1, \mathbf{x}_2)^2
$$
with a metric $d(\mathbf{x}_1, \mathbf{x}_2)$

\bibliographystyle{acm} % We choose the "plain" reference style
\bibliography{citation.bib} % Entries are in the refs.bib file
\end{document}
